% This must be in the first 5 lines to tell arXiv to use pdfLaTeX, which is strongly recommended.
\pdfoutput=1
% In particular, the hyperref package requires pdfLaTeX in order to break URLs across lines.

\documentclass[11pt]{article}

% Remove the "review" option to generate the final version.
\usepackage[review]{acl}

% Standard package includes
\usepackage{times}
\usepackage{latexsym}

% For proper rendering and hyphenation of words containing Latin characters (including in bib files)
\usepackage[T1]{fontenc}
% For Vietnamese characters
% \usepackage[T5]{fontenc}
% See https://www.latex-project.org/help/documentation/encguide.pdf for other character sets

% This assumes your files are encoded as UTF8
\usepackage[utf8]{inputenc}

% This is not strictly necessary, and may be commented out,
% but it will improve the layout of the manuscript,
% and will typically save some space.
\usepackage{microtype}

% If the title and author information does not fit in the area allocated, uncomment the following
%
%\setlength\titlebox{<dim>}
%
% and set <dim> to something 5cm or larger.

\title{Sarcasm Classification}

\author{Amy Tzu-Yu Chen \\
  University of Washington \\
  \texttt{amy17519@uw.edu} \\\And
  {David Roesler \\
  University of Washington \\
  \texttt{droesl@uw.edu} \\} \\\AND
  {Diana Baumgartner-Zhang \\
  University of Washington \\
  \texttt{diazhang@uw.edu} \\} \\\And
  Juliana McCauslabd \\
  University of Washington \\
  \texttt{jumc1469@uw.edu} \\}


\begin{document}

\maketitle

\begin{abstract}

TBD.

\end{abstract}

\section{Introduction}

[DRAFT]
In any text format, detecting whether someone is sarcastic or serious can be challenging. For spoken interactions, we can rely on intonation, facial expressions, and other non-verbal cues to inform us on the underlying meanings. On Twitter or other social media platforms, users often express their opinions or thoughts solely through text. Even for other people, discerning whether a text should be interpreted literally or figuratively can prove to be a challenge.

Sarcasm differs from irony in that it is mocking and often critical in nature. Often, the underlying sentiment is negative, but appears positive on the surface.

The use of sarcasm on the internet is ubiquitous, and its presence can disrupt computational systems of sentiment analysis \citealp{Liu2010SentimentAA}.

\section{Task Description}

Our primary task is sarcasm detection in English Twitter text, which we treat as a binary classification problem. Given a text, our task is to determine whether it is sarcastic or non-sarcastic. Sarcasm is a form of verbal irony through which a speaker expresses their stance toward a topic, which often takes the form of contempt or derogation \citealp{WILSON20061722}. Automatic sarcasm detection \citealp{joshi:automatic} is the prediction of the presence of sarcasm in text. 
Twitter, a platform often used to express the critical viewpoints of its users, has been a common data source for sarcasm detection models \citealp{doi:10.1177/1470785320921779}. To train and evaluate our model, we make use of the Twitter sarcasm dataset from SemEval 2022 Task 6, iSarcasmEval \citealp{oprea-magdy-2020-isarcasm}. Unlike sarcasm datasets labeled by third-party annotators, this dataset contains labels provided by the authors of the tweets themselves. The iSarcasmEval data includes both English and Arabic sets. 

As our secondary adaptation task, we perform sentiment detection on the Arabic portion of the dataset. To evaluate the performance of our binary classification model, we measure F1 score on the positive (sarcastic) class.

\section{System Overview}

TBD.

\section{Approach}

TBD.

\section{Results}

TBD.

\section{Conclusion}

TBD.

% Entries for the entire Anthology, followed by custom entries
\bibliography{anthology,custom}
\bibliographystyle{acl_natbib}

\appendix

\end{document}
